
\section{Pattern MVC}
\subsection{Modèle écoutable}
Un modèle écoutable est une classe abstraite permettant d'avoir un modèle qui possède des écouteurs et les prévient lorsqu'il se met à jour. Il est composé de plusieurs méthodes telles que :

\begin{itemize}
\item{\textbf{ajoutEcouteur}}
Permet de rajouter des écouteurs au modèle.
\item{\textbf{retraitEcouteur}}
Permet de retirer des écouteurs.
\item{\textbf{changement}}
Permet de prévenir tout les écouteurs d'un changement au modèle.
\end{itemize}

Le modèle écoutable est implémenté par la classe bataille
\subsection{Vue}
\subsubsection{Ecouteur}
L'interface écouteur est lié au modèle écoutable expliqué précédemment,
elle ne contient qu'une méthode \textbf{miseAJour}(Object o) qui est 
appelée par le modèle écoutable, cette méthode permet aussi de recevoir 
un objet afin de déterminer quel modèle lui envoie une requête de mise 
a jour.
\subsubsection{Vue}
La vue est une implémentation de l'interface écouteur, elle permet
d'afficher le modèle et de s'actualiser lorsque le modèle se met a jour.
Dans notre projet la vue est implémentée par les classes \textbf{vueTerminal} et \textbf{vueInterface}, afin de pouvoir afficher notre bataille navale sur le terminal et sur une interface graphique swing.
\subsection{Contrôleur}
Les contrôleurs permettent d'effectuer des actions sur le modèle et donc
de le mettre indirectement a jour. Il est donc nécessaire de créer des
contrôleurs adapté a la vue que sont \textbf{controleurTerminal} et \textbf{controleurInterface}, ainsi qu'un troisième contrôleur: \textbf{controleurBateau} qui permet de placer les bateaux dans l'interface graphique.