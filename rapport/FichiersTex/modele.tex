
\section{Modèle}

\par Lors de l'instanciation du modèle 2 joueurs sont créés. Par défaut le joueur 1 sera toujours un humain et le joueur 2 soit un RandomPlayer ou un IA Player, mais il est possible de faire s'affronter 2 Humains ou 2 Ordinateurs. Ces joueurs peuvent d'interagir avec les plateaux grâce à la fonction \textbf{attackP}(Coord c,Joueur jAttaquant,Joueur jAttaque), qui permet à un joueur d'attaquer le plateau de son adversaire.

\subsection{Les joueurs}

Il existe 3 types de joueurs :
\\
\begin{itemize}
\item{Humain}
\item{RandomPlayer}
\item{IAPlayer}\\
\end{itemize}

\par Le RandomPlayer est un joueur qui effectue des coups de manière aléatoire contrairement à l'IA Player, qui a des stratégies d'implémentées. L'Humain est évidemment entièrement contrôlé par la personne qui joue au jeu.
\par Chaque joueur possède un plateau qui lui est propre.

\subsubsection{Plateau}

L'objet \textbf{Plateau} représente la grille de jeu d'un joueur. Un plateau est un tableau bi-dimensionnel de taille 10 par 10 (taille par défaut). Un plateau est rempli par le biais d'une enum "StateShot" :
\\
\begin{itemize}
\item{WATER}
\item{SHIP}
\item{WATER\_HIT}
\item{SHIP\_HIT}\\
\end{itemize}

\par Quand un plateau est créé il est par défaut rempli d'eau (StateShot.WATER). Un plateau possède plusieurs méthodes utiles au déroulement de la partie :  
\\
\begin{itemize}
\item{\textbf{ajoutBateau(Bateau b) :}} ajoute le bateau passé en paramètre dans le plateau.
\item{\textbf{plusDeBateaux() :}} permet de savoir si les bateaux d'un joueur sont tous coulés.\\
\end{itemize}
Un plateau possède une liste contenant tous ses bateaux.

\subsubsection{Les bateaux}
\par L'objet \textbf{Bateau} représente un des bateau d'un plateau, et est caractérisé par :\\

\begin{itemize}
\item{Sa taille}
\item{Son orientation}
\item{Sa coordonnée de départ}\\
\end{itemize}

Un joueur possède un maximum de 5 bateaux :\\

\begin{itemize}
\item{Un bateau de taille 5 (le porte-avions)}
\item{Un bateau de taille 4 (le croiseur)}
\item{Deux bateau de taille 3 (les sous-marin)}
\item{Un bateau de taille 2 (le torpilleur)}\\
\end{itemize}

L'orientation d'un bateau est représenté par une enum "Orientation" : \\

\begin{itemize}
\item{BAS}
\item{DROITE}\\
\end{itemize}

Un bateau possède plusieurs méthodes :\\

\begin{itemize}
\item{\textbf{isCoordInside(Coord c) :}} indique si une coordonnée est parmi les coordonnée que couvre le bateau, ce qui est utile pour ne pas superposer des bateaux.
\item{\textbf{isCoordNear(Coord c) :}} indique si une coordonnée est adjacente orthogonalement au bateau, ce qui permet d'éviter d'avoir 2 bateaux qui se touchent.
\item{\textbf{isInPlateau(Plateau p) :}} permet de savoir si le bateau créé est dans les limites du plateau.
\item{\textbf{isSunk(Plateau p) :}} permet de savoir si le bateau est coulé.
\end{itemize}